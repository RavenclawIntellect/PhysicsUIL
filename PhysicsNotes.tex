\documentclass{article}

\usepackage{TeachFormat}
\usepackage{multicol}
\usepackage{mdframed}
\usepackage[version=4]{mhchem}
\usepackage{array}
\usepackage{tabularx}
\usepackage[thinlines]{easytable}

\title{UIL Physics Notes}
\author{Aaron Zhou}

\renewcommand{\thefootnote}{\fnsymbol{footnote}}

\begin{document}

\maketitle

{
    \hypersetup{linkcolor=black}
    \fontfamily{lmss}
    \tableofcontents
}

\newpage
\section{DC Circuits}

We use the following notation for the section: 

\vspace{15px}

\begin{tcolorbox}
\begin{multicols}{4}
\begin{description}
    \item $\emf$ = emf
    \item $I$ = current
    \item $P$ = power
    \item $q$ = charge
    \item $R$ = resistance
    \item $\Delta V$ = voltage
    \item $W$ = work
\end{description}
\end{multicols}
\end{tcolorbox}

\begin{textbox}{Author's Note}
    For this section, we adopt the conventional current standard of circuits (i.e. positive carriers carry charge along a circuit). For equations that determine the direction of current, use discretion to determine the direction of current. 
\end{textbox}

\subsection{Introduction}
\vspace{10px}
We begin by defining the electromotive force, also denoted $\emf$. It is defined as the work per unit charge:
\begin{equation}
    \emf = W/q = dW/dq.
\end{equation}
It is important to note that the electromotive force is not an actual force; rather, it is a potential difference that causes charge to move. Thus, it is important to think of the electromotive force as something that \textit{motivates} charge to move rather than an actual force causing charge to move. 

\subsection{Voltage Equations}

Since the electromotive force can be converted into a voltage potential, to properly perform computations in DC circuits, we need to perform calculations based on voltage. Thus, we introduce the following theorems regarding voltage:

\begin{thm}{Kirchoff's Loop Rule}
    The sum of all differences in potential around a complete circuit loop must be zero.
\end{thm}

\begin{thm}{Kirchoff's Junction Rule}
    The sum of the currents entering a junction is equal to the sum of the currents leaving the junction. 
\end{thm}

\begin{thm}{Voltage as a State Function}
    Voltage is \textit{path-dependent}; that is, voltage difference between two points does not depend on the path that the voltage takes. 
\end{thm}

\vspace{10px}
We also have one law that essentially acts as our bread and butter when it comes to working with DC circuits, Ohm's Law:

\begin{eq}
    \Delta V = IR. 
\end{eq}

\newpage
Now, given that voltage is path-independent, we can combine Theorems \csecthm{2}, \csecthm{3}, and \cseceqn{5} to get the following important theorem for all DC circuits:

\begin{thm}{Resistors in Series and Parallel}
    Consider $n$ resistors with resistances $R_1, R_2, \dots, R_n$. If they are placed in a series, then the equivalent resistance is
    \[R_{\textrm{eq}} = \sum_{n} R_n.\]
    If they are placed in parallel, then the equivalent resistance satisfies 
    \[\frac{1}{R_{\textrm{eq}}} = \sum_n \frac{1}{R_n}.\] 
\end{thm}

\newpage
\section{Special Relativity}

\subsection{Introduction}

Relativity is a modern physics idea that relates space and time together while considering objects with particularly large mass or particularly large speed. It is based off of two basic ideas:

\begin{defi}{Basic Ideas of Relativity}
    \begin{enumerate}
        \item The laws of physics are invariant in all inertial frames of reference (that is, frames with no acceleration). 
        \item The speed of light in vacuum is the same for all observers, regardless of the motion of light source or observer. 
    \end{enumerate}
\end{defi}

\vspace{10px}
From these basic ideas, we can derive the theory of relativity. We will begin with special relativity. 

\subsection{Special Relativity Equations}

Since relativity is inherently related to the speed of light, we need a way to determine how "close" something is to relativistic speeds, or the speed of light. This is the definition of the Lorentz factor:

\begin{eq}
    \lambda = \frac{1}{\sqrt{1 - \frac{v^2}{c^2}}}
\end{eq}

Notice that we always have the Lorentz factor greater than or equal to 1. It is important to note that many of our basic definitions of energy and time break down when moving at relativistic speeds. Thus, we must rederive many of our existing equations. First, we have the most famous two effects in special relativity: length contraction and time dilation. At relativistic speeds\footnote{The term \textit{relativistic speeds} will denote any speed in which the effects of special relativity are substantial; generally, this will be at velocities $v > c/10$; however, the equations can be applied to any situation where the effect s of special relativity must be considered.}, distances seem to contract in the direction of the movement of the observer's reference frame, while time seems to dilate during movement. Both length contraction and time dilation are scaled by the Lorentz factor, resulting in the two relations 
\begin{eq}
    L = \frac{L_0}{\lambda}
\end{eq}
\begin{eq}
    t = t_0\lambda. 
\end{eq}

Momentum follows a similar effect at relativistic speeds, similarly scaling by the Lorentz factor:

\begin{eq}
    p = \lambda mv. 
\end{eq}

We now consider the energy of a particle moving at relativistic speeds. There are two types of energy to consider: the rest energy of a particle and its kinetic energy. The rest energy of a particle is the energy the particle contains within its mass while not moving, while the kinetic energy of a particle is the energy it gains from its movement. The equations for these two values are given respectively by:

\begin{eq}
    E_0 = mc^2
\end{eq}
\begin{eq}
    K = (\lambda - 1)mc^2.
\end{eq}

Now, we can combine Equations \cseceqn{5} and \cseceqn{6} by summing the two equations together to get the total energy of the particle:

\begin{eq}
E = E_0 + K = \lambda mc^2.
\end{eq}

Given each of the above relativistic equations, it should now be possible for each of the quantities from classical mechanics to be rederived in terms of special relativity\footnote{Many of these derivations do require calculus (e.g. the derivation of force requires the derivative of relativistic linear momentum with respect to time), so we advise caution in blatantly applying classical mechanics to special relativity.}. We will leave these derivations as an exercise to the reader. 

\newpage
\section{Nuclear Physics}

\subsection{Introduction}
Most of our investigation of nuclear physics will arise from analyzing nuclear reactions: specifically, we care about what types of nuclear reactions can occur and how to classify them. We will begin by analyzing basic reaction types.

\subsection{Introductory Reactions}

There are two basic flavors of nuclear reactions that are important to know: the alpha decay and the beta decay. Alpha decay occurs at the emission of a Helium-4 nucleus from an atom, as seen in Equation \cseceqn{1}:

\begin{eq}
    \ce{^{238}U} \rightarrow \ce{^{234}Th} + \ce{^{4}He}. 
\end{eq}

In contrast, beta decay occurs at the conversion of a neutron to a proton or vice versa, resulting in the creation of an electron or a positron. Beta decay that creates an electron is called $\beta^-$ decay, while beta decay that creates a positron is called $\beta^+$ decay. The basic reaction for the former is shown in Equation \cseceqn{2}, while the basic reaction for the latter is shown in Equation \cseceqn{3}:

\begin{eq}
    \ce{n} \rightarrow \ce{p} + \ce{e-} + \bar v
\end{eq}
\begin{eq}
    \ce{p} \rightarrow \ce{n} + \ce{e+} + v
\end{eq}

In both of these equations, either $\bar v$ or $v$ showed up as a result of the decay; we wil note that these are the antineutrino and neutrino, respectively. For now, we will not elaborate on the production of this neutrino further. 

\newpage
\subsection{The Four Fundamental Forces}

Before we can go deeper into examining nuclear and particle reactions, we must first discuss the forces that mediate such interactions. These are the four fundamental forces, described below:

\begin{defi}{Four Fundamental Forces}
    The four fundamental forces are:
    \begin{enumerate}
        \item \textit{Gravity.} Gravity is the weakest out of the four fundamental forces; as such, on a particle by particle level, gravity will be negligible compared to the other three forces. Almost no particle interactions are mediated by the gravitational force. Reactions mediated by this take years. 
        \item \textit{The weak force.} The weak force is responsible for beta decay and other processes involving fundamental particles. However, due to its low relative strength, it has a range of less than 1 fm and is significantly less powerful than the strong force. Reactions mediated by this take between $10^{-13}$ and $10^{-8}$ seconds.
        \item \textit{The electromagnetic force.} The electromagnetic force is the interaction of charges within different particles. Interactions can occur due to the electromagnetic force, as it has an infinite range, and many common macroscopic forces such as drag and friction are the result of the electromagnetic force. The electromagnetic force is relatively strong enough to be critical in determining the structure and stability of nuclei. Reactions mediated by this take between $10^{-20}$ and $10^{-14}$ seconds. 
        \item \textit{The strong force.} The strong force is responsible for the binding of nuclei, and is dominant in many of the reactions and decays between fundamental particles. However, its scope is limited both in that some particles can not feel its effects, and that it has a relatively short range of around 1 fm. Reactions mediated by this take around $10^{-23}$ seconds.
    \end{enumerate}
\end{defi}

\subsection{Families of Particles}

The final step we must perform before we can finally analyze particle reactions in full is to understand families of particles. Particles differ widely in their properties, so we must separate them into some general groups to allow for us to get some grasp on the mechanisms of nuclear reactions. 

\vspace{10px}
The first category of particles we analyze is the leptons. Leptons are characterized by all known leptons having a spin of $1/2$. Interesting, this entire group of particles can only interact through weak and electromagnetic interactions, as none of them are affected by the strong force. The complete list of leptons and their antiparticles can be seen in the below table:

\begin{center}
    \begin{tabular}{ |>{\centering\arraybackslash}m{12em}|>{\centering\arraybackslash}m{12em}| } 
     \hline
     Particle & Antiparticle \\ 
     \hline
     \ce{e-} & \ce{e+} \\ 
     \hline
     $v_\textrm{e}$ & $\bar v_\textrm{e}$ \\ 
     \hline 
     $\mu^-$ & $\mu^+$ \\ 
     \hline 
     $v_\mu$ & $\bar v_\mu$ \\ 
     \hline 
     $\tau^-$ & $\tau^+$ \\ 
     \hline 
     $v_\tau$ & $\bar v_\tau$ \\ 
     \hline
    \end{tabular}
\end{center}

Note that there are three charged particles paired with an uncharged neutrino. Although the electron is a stable particle, the meson and tau decay into other leptons:

\begin{eq}
    \mu \rightarrow \ce{e-} + \bar v_\textrm{e} + v_\mu
\end{eq}
\begin{eq}
    \tau \rightarrow \mu^- + \bar v_\mu + v_\tau
\end{eq}
We should note that both of the above reactions must be mediated by the weak force, as any reaction involving neutrinos will be mediated by the weak interaction.

\vspace{20px}

The second family of particles we will analyze is the mesons. mesons are strongly interacting particles with integral spin. The most common examples of mesons are the pion and Kaon, which are denoted $\pi$ and K, respectively. Generally, mesons are produced in reactions mediated by the strong interaction and will decay into other mesons and leptons in reactions mediated through the strong, electromagnetic, or weak interaction. meson production can be seen in Equation \cseceqn{7}, while meson decay can be seen mediated by the weak interaction and electromagnetic interaction, respectively, in Equations \cseceqn{8} and \cseceqn{9}. 

\begin{eq}
    \ce{p} + \ce{n} \rightarrow \ce{p} + \ce{p} + \pi^-
\end{eq}

\begin{eq}
    \pi^- \rightarrow \mu^- + \bar v_\mu 
\end{eq}

\begin{eq}
    \pi^0 \rightarrow \gamma + \gamma
\end{eq}

\vspace{20px}
The third class of particles that we will analyze are the baryons. Baryons have half-integral spins. Notable examples of baryons are the proton and neutron. Baryons will have distinct antiparticles; for example, the antiproton $(\bar p)$ and the antineutron $(\bar n)$. Interactions between nucleons (protons and neutrons) can produce heavier baryons, such as the reaction shown in Equation \cseceqn{10}:

\begin{eq}
    \ce{p} + \ce{p} \rightarrow \ce{p} + \Lambda^0 + \ce{K+}
\end{eq}

In the above equation, $\Lambda^0$ is the desired heavy baryon. 

\vspace*{10px}
The final class of particles we will analyze are the field particles, also known as the exchange particles. These particles essentially act as the mediators for the forces that govern each reaction; these particles establish the fields the forces are generated from. Each force has its characteristic field particles: the weak bosons, $W^+, W^-, Z^0$, mediate the weak force, while the photon $\gamma$, mediates the electromagnetic force, and the gluon $\ce{g}$, mediates the strong force. The presence of any of these particles determines the type of interaction a reaction is mediated by, which can provide useful information regarding what can and can not occur. 

\vspace*{10px}
We also note that field particles have a slight contradiction to them: Each of the particles are emitted by the reactants in a nuclear reaction to form the force, yet each of the particles also has a nonzero mass, so how can conservation of mass be satisfied? This comes down to the idea of uncertainty: essentially, if we measure the energy of a reaction over some given time interval $\Delta t$, we are met with a corresponding uncertainty in the energy of reaction by the Heisenberg Uncertainty Principle, shown in  Equation \cseceqn{11}:

\begin{eq}
    \Delta E \geq \frac{h}{2\pi \Delta t}.
\end{eq}

Thus, if we measure for a small enough interval of time, we are able to neglect the energy loss from the emission of a field particle due to uncertainty. Interestingly, this is what defines the range of each of the fundamental forces; because we can only observe a reaction for a given amount of time, a field particle will only be able to move a finite distance during the course of the reaction, so the reaction has a maximum range. 

\newpage

\subsection{Conservation within Reactions}

With the fundamentals of nuclear physics established, we are finally able to describe which nuclear reactions are possible to occur:

\begin{thm}{Reaction Mechanisms}
    In any particle or nuclear reaction, the following three properties must be satisfied:

    \begin{enumerate}
        \item \textit{Conservation of Momentum:} If a single particle is decaying, then it must decay into at least two other particles. 
        \item \textit{Conservation of Charge:} The total charge of the reactants must equal the total charge of the products.
        \item \textit{Conservation of Baryon Count}: The baryon counts in the reactants and products must be the same. 
        \item \textit{Conservation of Lepton Count}: The three lepton counts in the reactants and products must be the same. 
        \item \textit{Conservation of Strangeness}: In any reaction mediated by the electromagnetic or strong interactions, total strangeness must be conserved between products and reactants. In any reaction mediated by the weak interaction, strangeness is allowed to change by at most 1 between products and reactants.
        \item \textit{Conservation of Angular Momentum:} Spin must be conserved between the products and reactants. 
    \end{enumerate}
\end{thm}

\vspace*{10px}
Points 1-3 of Theorem \csecthm{12} are clearly explanatory\footnote{We should note that the baryon count is decreased by baryon antiparticles; the theorem is slightly deceiving in that regard.}; we will focus on Points 4-6. We will begin with strangeness. The \textit{strangeness} of a particle is a property of that particle that determines its behavior within different particle reactions. Unfortunately, unlike baryon or lepton count, the strangeness of a particle is something intrinsic to each particle. We will provide some of the most common particles and their strangenesses in a later table. 

\vspace{10px}
We now look at the idea of lepton count in Theorem \csecthm{12}. The point references three lepton numbers. These are $L_e$, $L_\mu$, and $L_\tau$, which count the number of leptons of each of the three flavors of lepton. \textbf{Each} of these must be conserved (e.g. a $\mu$ lepton can not be converted to a $\tau$ lepton). We note that importantly, lepton count is \textit{decreased} by their antiparticles; thus, $\bar v_e$ contributes $-1$ to the $L_e$ lepton count.

\vspace{10px}
Finally, the last element of particle reactions is up to the idea of angular momentum and spin. We know that angular momentum is conserved in a nuclear reaction, which means that the intrinsic property of spin must be conserved as well. However, this is all we will say on the matter, as actual spin calculations seen to be out of the scope of a typical high school course. 

\newpage
\subsection{Quarks}

On top of our discussion of nuclear particle reactions, we must discuss the interactions between particles even more fundamental than the baryons and mesons: the quarks. The quarks are fundamental particles that come in 6 "flavors": up (u), down (d), strange (s), charm (c), top (t), and bottom (b).

The key properties of the six quarks are shown in the below table:\footnote{TODO: I should probably redo this table using the easytable package. }

\begin{center}
    \begin{tabular}{ >{\centering\arraybackslash}m{6em} >{\centering\arraybackslash}m{6em} >{\centering\arraybackslash}m{6em} >{\centering\arraybackslash}m{6em} >{\centering\arraybackslash}m{6em} } 
     Quark & Charge & Spin & Baryon Number & Strangeness \\ 
     \hline
     Up (u) & 2/3 & 1/2 & 1/3 & 0 \\ 
     Down (d) & -1/3 & 1/2 & 1/3 & 0 \\
     Charm (c) & 2/3 & 1/2 & 1/3 & 0 \\ 
     Strange (s) & -1/3 & 1/2 & 1/3 & -1 \\ 
     Top (t) & 2/3 & 1/2 & 1/3 & 0 \\ 
     Bottom (b) & -1/3 & 1/2 & 1/3 & 0 \\ 
    \end{tabular}
\end{center}

\vspace*{10px}
Quarks are found in baryons and mesons, where they serve as the fundamental particles that make baryons and mesons. In fact, the properties of many baryons and mesons are determined by the quark configuration, as simply adding up the values for the spin, baryon number, and charge of the constituent quarks for any given particle will result in the spin, baryon number, and charge of the particle itself. (In this vein, it is important to note that each quark has an antiquark with opposing charge and baryon number, which is how mesons have a baryon number of 0.) The quark configurations of several important particles are shown below, from which this property can be verified. 

\vspace*{10px}

\begin{center}
    \begin{TAB}(r, .3cm, .5cm)[5pt, 12cm, 0cm]{|c|c|}{|c|c|c|c|c|c|c|}
     Particle & Quark Configuration \\ 
     p & uud \\
     n & udd \\
     $\pi^+$ & $u\bar d$ \\
     $\pi^0$ & $\bar u u $, $\bar d d$ \\
     $\ce{K^0}$ & $d\bar s$ \\ 
     $\ce{K^-}$ & $\bar u s$ \\
    \end{TAB}
\end{center}

\vspace*{10px}

In nuclear reactions, the number of each flavor of quark is conserved \textit{if the reaction is mediated by the electromagnetic or strong forces}. This in fact gives us a method to identify reactions mediated by the weak interaction; if a nuclear reaction has a flavor change in one of its quarks between products and reactants, then it must be mediated by the weak interaction. (Note that neither the converse nor the inverse of this statement are necessarily true.) We will call such reactions \textit{flavor-changing reactions.}

\newpage
\section{AC Circuits}

\subsection{Introduction}

In this section, we introduce the concept of alternating current. In circuits with alternating current, the voltage-providing element does not provide a constant voltage, but rather a voltage that varies sinusoidally. We will investigate some of the properties of such circuits in this section. 

\subsection{Voltage Producing Elements}

We begin by focusing on the voltage producing element. We define the \textit{peak voltage} of the circuit as the maximum voltage produced by the voltage producer. We also define the \textit{rms voltage} of the circuit, which is the average voltage of the circuit (computed by essentially taking a quadratic mean). The peak voltage and the rms voltage are related by the equation 
\begin{eq}
    V_{\textrm{rms}} = \frac{V_0}{\sqrt{2}}.
\end{eq}

We note that the root mean voltage can be calculated for a continuous waveform as well. Let one of the periods of the voltage be $t_1 \leq t \leq t_2$. Then, the rms voltage can be calculated as 
\begin{eq}
    V_{\textrm{rms}} = \sqrt{\frac{1}{t_2 - t_1}\int_{t_1}^{t_2} V(t)^2 \ dt}.
\end{eq}

We note that we can define current similarly as well. Using the same definition of peak current and rms current, we can relate the two values by the equation 
\begin{eq}
    I_{\textrm{rms}} = \frac{I_0}{\sqrt{2}}. 
\end{eq}
In an AC circuit, we can assume that the voltage produced gives rise to a current 
\begin{eq}
    I = I_0\cos 2\pi ft,
\end{eq}
where $f$ is the frequency of the sinusoidal voltage and $t$ is the time elapsed.

\subsection{Circuit Elements}

We now need to consider what occurs when our source of alternating voltage is placed with other circuit elements. We will begin with resistors. Since resistors have a constant resistance, we can use Ohm's law to derive the voltage across the resistor:
\begin{eq}
    V = IR = I_0\cos 2\pi ft. 
\end{eq}

As a result, we can say that a resistor is \textit{in phase} with the voltage-producing element, as the graphs of the current and the voltage across the resistor have the same shift at any given time. Additionally, because of this, the power equations for resistors can hold up if we consider the average power dissipated and the rms current and voltages:
\begin{eq}
    \bar P = I_{\textrm{rms}}^2R = \frac{V_\textrm{rms}^2}{R}. 
\end{eq}

We now consider inductors within AC circuits. We recall that the counter-emf produced by an inductor has value 
\begin{eq}
    V = L\frac{dI}{dt}. 
\end{eq}

\vspace{10px}
By applying the differential, we can derive the equation across an inductor to be 
\begin{eq}
    V = -V_0\sin 2\pi ft.
\end{eq}

As we see above, we see that since the voltage and current are governed by sine and cosine waves, respectively, the current lags behind the voltage by $90\dg$ in an inductor. We note that since an inductor impedes the flow of charge similarly to a resistor, we can use an equation similar to that of a resistor. However, since peak voltage and peak current are not reached at the same time, \textbf{such an equation can only be used for rms or peak voltages and currents.} This equation is the reactance equation, where we have 
\begin{eq}
    V = IX_L,
\end{eq}
where $X_L$ is defined as the \textit{inductive reactance} of the circuit. The inductive reactance of a circuit can be found using 
\begin{eq}
    X_L = \omega L = 2\pi fL.
\end{eq}
We note that in circuits containing inductors along with other circuit elements, the reactance equation should be used carefully: specifically, because the inductor is no longer the only component impeding voltage within the circuit, the reactance equation should only be used when the reactance is large compared to any other resistance. 

\vspace*{10px}
We now finish the section on circuit elements with considerations regarding capacitors. Using the fact that in a capacitor, $V = Q/C$ and that the capacitor acts like an open switch when it has charge and a wire when it does not, we can derive the fact that when a capacitor has charge, it has a high voltage and a low current. Thus, we can state that current leads voltage by  $90\dg$ in a capacitor. Equivalently, in an AC circuit with only a capacitor, we write 
\begin{eq}
    V = V_0\sin 2\pi ft. 
\end{eq}

We can also derive a capacitive reactance equation, where we have 
\begin{eq}
    V = IX_C,
\end{eq}
where $V$ and $I$ are only defined at rms or peak values and the reactance $X_C$ is defined by 
\begin{eq}
    X_C = \frac{1}{\omega C} = \frac{1}{2\pi fC}.
\end{eq}

\vspace*{10px}
We can define the impedance of a circuit. Impedance affects LRC circuits. Essentially, it is a way of calculating the peak or rms voltage and currents across the circuit. Thus, for rms or peak voltages and currents, we have 
\begin{eq}
    V = IZ,
\end{eq}
where $Z$ is the impedance of the circuit. The impedance $Z$ can be defined by 
\begin{eq}
    Z = \sqrt{R^2 + (X_L - X_C)^2}.
\end{eq}

Notably, the impedance of a circuit is incredibly helpful, as it allows us to find the peak or rms current. This can all be combined using the three equations \cseceqn{5}, \cseceqn{9}, and \cseceqn{12} to find the individual voltages across each of the inductor, capacitor, and resistor. This creates a good way of finishing an LRC circuit.

\vspace*{10px}
It is important to note that there are also two other times that a circuit involving resistors, capacitors, and inductors can be analyzed: at $t = 0$ and at $t = \infty$. At $t = 0$, since capacitors have no charge, they act as wires; similarly, inductors will act as open switches. At $t = \infty$, also known as steady-state, capacitors gaining charge will act as open switches while inductors will act as wires, unable to further produce a counter-emf. However, it is notable that such a modification will only occur in DC circuits, but we add it here due to this being the only discussion location for all three circuit elements. 

\newpage

\section{Light}

\subsection{Introduction}

This section will serve as a basis for all interactions involving light: its bending through lenses and mirrors, diffraction, refraction, etc. This will be more of a formula disposal location rather than an in depth explanation, however. 

\subsection{Lenses and Mirrors}

We will begin with the basis of much of optics: the interaction of light with lenses and mirrors. We will begin by focusing on a spherical mirror. A spherical mirror is defined by a center of curvature and a radius. When incident parallel rays interact with a mirror, they will all reach \textit{approximately}\footnote{The rays do not actually meet at the same focal point. The more curved a mirror is, the more spread out the points of contact of the rays of light will be. This effect is called spherical aberration.} the same point. This point is known as the focal point. The focal point sits at the midpoint of the center of curvature and the mirror, as denoted by the equation
\begin{eq}
    f = \frac{r}{2}.
\end{eq}

We note that the equation above does neglect the sign of the focal length; we will correct this when we get to an equation that requires it. 

\vspace*{10px}
When considering the effect a mirror has on light, we care about two different things: the height of the image $h_i$, and the distance from the mirror the image appears to be $d_i$. First, due to similar triangles, we have the simple equation\footnote{This equation lacks appropriate signs; the absolute values of both quantities should be equal.}
\begin{eq}
    \frac{h_o}{h_i} = \frac{d_o}{d_i}.
\end{eq}

Then, we can do some derivations with ray tracing (which we will not show here) to derive the all-important \textbf{mirror equation:}
\begin{eq}
    \frac{1}{d_i} + \frac{1}{d_o} = \frac{1}{f}.
\end{eq}

The mirror equation tends to be quite important, as it is the primary way the properties of an image can be deduced from only information about the original object and mirror. However, the mirror equation introduces an important issue: signs. Sign conventions are required for all lens and mirror applications, as handling each of the cases can be cumbersome without them. Thus, we establish the following sign conventions for mirrors:

\begin{thm}{Sign Conventions for Mirrors}
    \begin{enumerate}
        \item Let the object that is being reflected sit on the \textit{reflecting side of the mirror}. The distance from an object to a mirror is positive if the object is on the reflecting side of the mirror, and negative otherwise. Note that this also applies to the center of curvature, allowing the focal length to be negative. 
        \item The image height of an object is positive if the object is upright, and negative otherwise. Assume that the height of the original object is always positive. 
    \end{enumerate}
\end{thm}

After the mirror equation, we have one more piece of information we can extract about the mirror, also known as its magnification. The magnification of a mirror is a direct value of how much a mirror has enlarged and shrunk an object, defined by the signed equation 
\begin{eq}
    m = \frac{h_i}{h_o} = -\frac{d_i}{d_o}.
\end{eq}

\vspace{10px}
With all of these tools, we can mostly describe the properties of an image produced by a mirror. We will introduce one last distinction: the difference between the real and virtual images produced by a mirror. Real images are actual points in which light rays converge, while virtual images are points in which the light rays diverge. For mirrors, real images occur when the image lies on the reflecting side of the mirror, while virtual images occur when the image lies on the nonreflecting side of the mirror.

\vspace*{10px}
We now move on to lenses. Lenses refract light to cause it to change direction as it passes through the lens. There are two primary types of lens: a converging lens, whose walls curve outwards, and a diverging lens, whose walls curve inwards. Luckily for us, equations \cseceqn{2}, \cseceqn{3}, and \cseceqn{5} all still hold true for lenses as they do for mirrors. However, due to the dual walls of the lens and its ability to refract, not reflect, we must change our sign conventions:

\begin{thm}{Sign Conventions for Lenses}
    \begin{enumerate}
        \item The focal length is positive for converging lens and negative for diverging lens.
        \item The object distance is positive if the object is on the side of the lens from which the light is coming; otherwise, it is negative. The object distance can only be negative if multiple lenses are used in combination.
        \item The image distance is positive if the image is on the opposite side of the lens from where the light is coming; if it is on the same side, the distance is negative. 
        \item The height of the image is positive if the image is upright and negative otherwise. The object height is always taken to be positive.
    \end{enumerate}
\end{thm}

\vspace{10px}
We now handle real and virtual images. For lenses, we state that the image of the lens is real if it is on the opposite side of the lens from the object and virtual otherwise. 

\vspace*{10px}
We now focus on one final piece of information with respect to lenses: the lensmaker's equation. The lensmaker's equation relates the focal length of a lens to the way and materials it was constructed from. In order to use the lensmaker's equation, we first must describe the construction of a lens. Typically, a thin lens can be described by 2 radii of curvature. We say that one side of the lens is convex if it arches outwards and concave if it arches inwards. (Whether a side arches outwards or inwards can be determined by imagining light passing through the lens.) For the lensmaker's equation, we have the sign convention of describing the radius of curvature as positive if the lens is convex and negative if the lens side is concave. Thus, we can now employ the formula

\begin{eq}
    \frac{1}{f} = (n - 1)\left(\frac{1}{R_1} + \frac{1}{R_2}\right),
\end{eq}

where $n$ is the index of refraction of the material the lens is made of. 

\vspace{10px}
We also have one more type of object used to manipulate the direction and movement of light: spherical refracting surfaces. We define spherical refracting surfaces by a radius of curvature $r$. For spherical refracting surfaces, we care only about the location where the object is and the location where the image appears. The equation for such devices are defined by the index of refraction of the medium containing the incident light $n_1$\footnote{It is important to note here that it is commonly used that incident or refracted light comes from air, which has an index of refraction of 1. Additionally, water has an index of refraction of 1.33.}, the index of refraction of the medium containing the refracted light $n_2$, and the distances of the object and its image from the refracting surface $o$ and $i$:

\begin{eq}
    \frac{n_1}{o} + \frac{n_2}{i} = \frac{n_2 - n_1}{r}.
\end{eq}

Of course, we must again establish sign conventions for such a refracting surface. We discuss this in the language of real and virtual images:

\begin{thm}{Sign Conventions for Spherical Refracting Surfaces}
    \begin{enumerate}
        \item We define the medium containing incident light as the $V$-side and the medium containing refracted light the $R$-side.  (This is how whether an image is real or virtual is defined.)
        \item If the center of curvature lies on the $R$-side, we call the radius of curvature positive; otherwise, we call the radius of curvature negative. 
        \item Object distances are positive if they are on the $V$-side. Image distances are positive if they are on the $R$-side. 
    \end{enumerate}
\end{thm}

\subsection{The Wave Nature of Light}

The fact that light acts both as a particle and as a wave gives it quite a few interesting properties we will explore here. We will begin with diffraction. 

\vspace{10px}
We begin by analyzing diffraction from the point of view of the double slit experiment. The famous Young's double slit experiment deduced that light acted as a wave by showing that light passing through a screen with two slits formed not a pattern of two slits, but a pattern of multiple slits. The pattern of slits over the screen was found to be a result of interference, where the places where light appeared were where constructive interference occurred, and the places where light was absent was where destructive interference occurred. We won't go too much into this mechanism for now, but we will note the two equations we have for calculating this behavior. We note that constructive interference occurs when 
\begin{eq}
    d\sin \theta = m\lambda,
\end{eq}
where $d$ is the distance between the slits, $\theta$ is the angle the rays of light must make with the horizontal, $m$ is an integer known as the order of the fringe, and $\lambda$ is the wavelength of light. Similarly, destructive interference occurs when 
\begin{eq}
    d\sin \theta = \left(m + \frac{1}{2}\right)\lambda. 
\end{eq}

\vspace{10px}
Next, we can investigate the properties of light passing through a single slit. Due to interference and diffraction (the reasons of which we won't clearly investigate), we again have multiple maxima and minima formed when light passes through a single slit. We note that the behavior is slightly different: the maxima of light occur at the values 
\begin{eq}
    D\sin \theta = m\lambda, \hspace*{15px}m = 0, \frac{3}{2}, \frac{5}{2}, \dots. 
\end{eq} 
Here, $D$ represents the width of the slit, and $\theta$ and $\lambda$ represent the same quantities they do in the double slit experiment. It is important to note that for a single slit, the maxima at $m = 0$ is significantly larger than any of the other maxima. Similarly, the minima of light occur at the values 
\begin{eq}
    D\sin \theta = m\lambda, \hspace*{15px}m = 1, 2, 3,\dots. 
\end{eq}

\vspace{10px}
We can now explore the extension of both the double and single slit versions of light diffraction: the interaction of light with a very large number of slits. Such a device is called a diffraction grating. Luckily for us, a diffraction grating produces the same minima and maxima as a double slit, described by equations \cseceqn{10} and \cseceqn{11}. However, there is an important difference between the double slit and the diffraction grating: the diffraction grating produces much sharper results. Since in the double slit version of the diffraction grating, there are only 2 light waves to perform interference with one another, the light spreads out across the maxima and minima. However, in a diffraction grating, any small disturbance will cause 100s of waves to be out of phase, causing much strong interference. Therefore, as a result, diffraction gratings have very strong maxima and are very weak elsewhere. 

\vspace*{20px}
We now turn our attention away from diffraction and now consider thin-film interference. Consider two surfaces, with the top surface being able to transmit light and having a differing index of refraction than the bottom surface. Upon light hitting the top surface, there are two possibilities for what occurs: either the light gets reflected off the top surface, or the light gets transmitted through the top surface and reflected off the bottom surface. This creates a path difference in the distance the light must travel, which causes a phase shift that leads to constructive and destructive interference. In order to correctly account for thin film interference, we must consider the following four rules:

\begin{thm}{Thin-Film Interference Rules}
    \begin{enumerate}
        \item The wavelength of a wave in a medium with index of refraction $n$ is $\lambda_n = \lambda / n$. 
        \item Constructive interference occurs when the path difference is an integer multiple of $\lambda_n$. 
        \item Destructive interference occurs when the path difference is a half-integer multiple of $\lambda_n$. 
        \item If a light wave in a medium with index of refraction $n_1$ reflects off a surface in a medium with index of refraction $n_2$, if $n_2 > n_1$, then a $180\dg$ phase change occurs. This is equivalent to adding $\frac{1}{2}\lambda_n$ for the path difference. 
    \end{enumerate}
\end{thm}

\end{document}